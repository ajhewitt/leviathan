\documentclass[11pt, a4paper]{article}
\usepackage[utf8]{inputenc}
\usepackage{geometry}
\usepackage{amsmath}
\usepackage{amssymb}
\usepackage[numbers]{natbib}
\usepackage{graphicx}
\usepackage{hyperref}

\geometry{margin=1in}

\title{\textbf{Project Leviathan: The Temporal Density Hypothesis}\\
\large Solving the Horizon Problem via Variable Causal Processing Rates}
\author{A. Hewitt \\ \small Principal Investigator}
\date{\today}

\begin{document}
\maketitle

\begin{abstract}
Standard cosmology ($\Lambda$CDM) assumes a linear flow of time ($dt = d\tau$) back to the Big Bang, requiring an ad-hoc ``Inflation'' field to solve the Horizon and Flatness problems. Project Leviathan proposes an alternative information-theoretic mechanism: the \textbf{Temporal Density Hypothesis}. We posit that the rate of causal processing (Effective Time, $\tau$) scales inversely with the complexity of the Universe's Phase Space, following a power law $\tau \propto t^{-\alpha}$. As $t \to 0$, the effective causal time approaches infinity, allowing the early Universe to structurally mature, homogenize, and select observer-compatible histories without requiring superluminal expansion. This project aims to falsify this hypothesis by auditing ``impossible'' early structures---massive high-z galaxies and super-horizon filaments---which serve as artifacts of this accelerated cosmic pre-history.
\end{abstract}

\section{Theoretical Framework: The Efficiency Curve}
We propose that the coordinate time $t$ (measured by atomic clocks) differs from the structural time $\tau$ (the accumulation of causal events). The relationship is governed by the \textit{Temporal Density} $\eta(t)$:

\begin{equation}
    d\tau = \eta(t) dt \quad \text{where} \quad \eta(t) \approx \left( \frac{t_0}{t} \right)^\alpha
\end{equation}

Here, $t_0$ is the current age of the Universe and $\alpha > 0$ is the \textit{Agenda Exponent}. If $\alpha = 0$, we recover Standard $\Lambda$CDM ($d\tau = dt$).

\subsection{The Singularity Asymptote}
The critical implication of this hypothesis arises as we approach the initial singularity ($t \to 0$). The total \textit{Effective Causal Time} $\tau_{total}$ experienced by the Universe since the Big Bang is the integral of the temporal density:

\begin{equation}
    \tau(t) = \int_{t}^{t_{now}} \eta(t') dt' \propto \int_{t}^{t_{now}} (t')^{-\alpha} dt'
\end{equation}

For any exponent $\alpha \ge 1$, this integral diverges:
\begin{equation}
    \lim_{t \to 0} \tau(t) = \infty
\end{equation}

\textbf{Physical Implication:} This implies that the ``first second'' of the Universe contained an infinite duration of causal processing. 
\begin{itemize}
    \item \textbf{Horizon Solution:} Regions of space that appear causally disconnected in linear coordinate time were in fact able to communicate and thermalize over an infinite structural history before expansion took over.
    \item \textbf{Information Selection:} This infinite pre-history allows the Universe to exhaustively search its phase space for a stable, observer-compatible configuration, replacing the random quantum fluctuations of Inflation with a deterministic selection process.
\end{itemize}

\section{Research Aims}

\subsection{Aim 1: The Chronometry Audit (Deriving $\alpha$)}
\textbf{Objective:} Empirically derive the exponent $\alpha$ by comparing the ``Structural Age'' of high-redshift objects against their allowed ``Coordinate Age.''
\\
\textbf{Methodology:}
\begin{itemize}
    \item Ingest JWST spectral data (CEERS/JADES) for galaxies at $z > 10$.
    \item Extract \textit{Stellar Population Ages} ($T_{struct}$) using spectral energy distribution (SED) fitting.
    \item Compare $T_{struct}$ vs. $\Lambda$CDM Age ($T_{coord}$).
    \item \textbf{The Curve Fit:} We fit the anomaly to the power law model $T_{struct} \approx T_{coord}^{(1-\alpha)}$ to find the best-fit $\alpha$.
\end{itemize}

\subsection{Aim 2: The Mega-Structure Audit (Horizon Violations)}
\textbf{Objective:} Audit the existence of structures exceeding the Homogeneity Scale ($> 370$ Mpc).
\\
\textbf{Methodology:}
\begin{itemize}
    \item Audit Quasar and GRB catalogs (SDSS, BOSS) for connected structures $> 1.2$ Gly (e.g., Hercules-Corona Borealis Great Wall).
    \item Such structures require a formation time $T_{form} \gg T_{coord}$. The discrepancy provides a secondary independent constraint on $\alpha$.
\end{itemize}

\subsection{Aim 3: The Void Audit (The Cold Spot)}
\textbf{Objective:} Test if the Eridanus Supervoid represents a region where ``vacuum clearing'' occurred at an accelerated rate.
\\
\textbf{Methodology:}
\begin{itemize}
    \item Cross-correlate Planck CMB maps with galaxy density maps.
    \item Determine if the void's depth and size ($R > 200$ Mpc) are statistically impossible ($>5\sigma$) in a standard Dark Energy growth model.
\end{itemize}

\section{Implications}
If $\alpha > 0$ is confirmed:
\begin{enumerate}
    \item \textbf{Inflation is Obsolete:} The smoothness of the CMB is explained by infinite causal contact in the ``pre-history'' near $t=0$.
    \item \textbf{JWST Anomalies Resolved:} Massive early galaxies are not impossible; they are simply older than their redshift suggests.
    \item \textbf{Teleology:} The Universe's initial conditions were not random, but the result of an exhaustive search for stable histories.
\end{enumerate}

\end{document}
